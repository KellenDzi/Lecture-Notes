\begin{presection}{Recall Berry phases from adiabatic evolution.
Think about gauge dependence vs gauge invariance.}

\begin{remarkbar}

Berry-phase quantities often look gauge-dependent at first glance,
but physical observables must be invariant.

\end{remarkbar}

\section{Berry Curvature}

\subsection{Berry Connection}

\begin{derivation}

\mathbf A_n(\mathbf k)
= i \langle u_n(\mathbf k) | \nabla_{\mathbf k} | u_n(\mathbf k) \rangle

\end{derivation}

\begin{equation}

\Omega_n(\mathbf k)
= \nabla_{\mathbf k} \times \mathbf A_n(\mathbf k)

\end{equation}

\begin{center}

\includegraphics[width=0.6\linewidth]{../figures/1.png}

\captionof{figure}{Berry curvature as an effective magnetic field in momentum space.}

\label{fig:F25-L5-1}

\end{center}

\begin{theorem}

The Berry curvature \(\Omega_n(\mathbf k)\) is gauge invariant.

\end{theorem}

\begin{lemma}

Under a gauge transformation \( |u_n\rangle \to e^{i\phi(\mathbf k)}|u_n\rangle \),
the Berry connection transforms as
\(\mathbf A_n \to \mathbf A_n + \nabla_{\mathbf k}\phi\).

\end{lemma}

\begin{corollary}

The curl of the Berry connection is unchanged under gauge transformations.

\end{corollary}

\begin{exercise}

Show explicitly that \(\Omega_n = \nabla_{\mathbf k} \times \mathbf A_n\)
is invariant under \( \mathbf A_n \to \mathbf A_n + \nabla_{\mathbf k}\phi \).

\end{exercise}

\begin{postsection}{Berry curvature plays the role of a magnetic field in momentum space.
Its integral over the Brillouin zone gives a Chern number.}
