\begin{presection}{Recall}
Recall the geometric phase from adiabatic evolution.
Connect Berry phases to parallel transport.

Think about why global phases are unphysical
but relative phases encode geometry.
\end{presection}

\begin{remarkbar}
Adiabatic evolution assumes a spectral gap.
Level crossings invalidate Berry-phase arguments.
\end{remarkbar}

\section{Geometric Phases in Quantum Mechanics}

\subsection{Adiabatic Theorem Review}

\begin{derivation}
\text{Consider a Hamiltonian $H(\lambda(t))$ with instantaneous eigenstates} \\
$H(\lambda)|n(\lambda)\rangle = E_n(\lambda)|n(\lambda)\rangle$. \\
\text{Assume slow evolution such that transitions are suppressed.}
\end{derivation}

\begin{equation}
|\psi(t)\rangle
=
e^{i\gamma_n(t)}
e^{-\frac{i}{\hbar}\int_0^t E_n(t')dt'}
|n(t)\rangle
\end{equation}

\begin{remarkbar}
The dynamical phase depends on time,
while the geometric phase depends only on the path.
\end{remarkbar}

\subsection{Berry Connection and Curvature}

\begin{derivation}
\mathbf A_n(\mathbf k)
=
i\langle u_n(\mathbf k)|\nabla_{\mathbf k}|u_n(\mathbf k)\rangle
E:
\Omega_n(\mathbf k)
=
\nabla_{\mathbf k}\times\mathbf A_n(\mathbf k)
\end{derivation}

\begin{center}
\includegraphics[width=0.6\linewidth]{figures/1.png}
\captionof{figure}{Berry curvature visualized as a magnetic field in momentum space.}
\label{fig:1}
\end{center}

\begin{center}
\includegraphics[width=0.6\linewidth]{figures/2.png}
\captionof{figure}{Parallel transport of quantum states around a closed loop in parameter space.}
\label{fig:2}
\end{center}

\begin{theorem}
The Berry curvature $\Omega_n(\mathbf k)$ is gauge invariant.
\end{theorem}

\begin{lemma}
Under a gauge transformation
$|u_n\rangle\to e^{i\phi(\mathbf k)}|u_n\rangle$,
the Berry connection transforms as
$\mathbf A_n\to \mathbf A_n+\nabla_{\mathbf k}\phi$.
\end{lemma}

\begin{corollary}
Physical observables depend only on $\Omega_n$,
not on the choice of gauge.
\end{corollary}

\begin{exercise}
Explicitly verify gauge invariance of $\Omega_n$
under $\mathbf A_n\to \mathbf A_n+\nabla_{\mathbf k}\phi$.
\end{exercise}

\begin{postsection}{Berry}
Berry curvature encodes global topology.
Its integral over the Brillouin zone yields a Chern number.

This topological invariant explains quantized Hall conductance
and protects edge states from local perturbations.
\end{postsection}

